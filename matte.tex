\documentclass[a4paper, 10pt]{article}
\usepackage[utf8x]{inputenc}
\usepackage[norsk]{babel}
\usepackage{natbib}
\usepackage{graphicx}
\usepackage[T1]{fontenc}
\usepackage{amsmath}
\usepackage{mathtools}

\title{MA0003 Cheatsheet}
\author{Kristoffer Dalby}
\date{}


\begin{document}

\maketitle

\thispagestyle{empty}
\newpage
\pagenumbering{arabic}
\setcounter{page}{1}

\section{Domain and Range}
Domain is the acceptable x-values and Range is the acceptable y-values.

\subsection{Domain}
$f(x) = \frac{6}{2-x}$\\
$2-x \neq 0 \quad \rightarrow \quad x \neq 2$\\
$D = (-\infty, 2) \cup (2, \infty)$ 
\quad or\\
$D = \{x | x \neq 2 \}$

\section{Slope and y-intercept}
Problem: $y-3x = 6$\\ 
Solve for y: $y=3x+6$\\
slope $\rightarrow 3x$ \quad y-int $\rightarrow 6$\\
$\text{slope} = m = 3$



\section{Derivative}

\subsection{Critical points}
The function:\\
$f(x) = y=x^2-6x+5$\\
The derivative:\\
$f'(x) = y'= 2x-6$\\
Set $y'$ to zero to find $x$\\
$y' = 0 \Rightarrow 0 = 2x-6 \Rightarrow x=3$\\
Solve for $x$\\
$y = (3^2) - (6 \cdot 3) + 5 \Rightarrow 9 - 18 + 5 = -4$\\
The critical point is $-4$

\subsection{Inflection points}
Find the second derivative of the function\\
$y'' = 2$\\
Set $y''$ to zero\\
$y'' = 0 \neq 2 \rightarrow \text{No inflection points}$

\subsection{Practical rules}
$\frac{d}{dx} (x^n)= nx^{n-1}$\\\\
$\frac{d}{dx} (f(x))^n) = n(f(x))^{n-1} \cdot f'(x)$\\\\
$\frac{d}{dx} (\sin x) = \cos x $\\\\
$\frac{d}{dx} (\sin f(x)) = \cos(f(x)) \cdot f'(x) $\\\\
$\frac{d}{dx} (\cos x) = -\sin x $\\\\
$\frac{d}{dx} (\cos f(x)) = -\sin(f(x)) \cdot f'(x) $\\\\
$\frac{d}{dx} (\tan x) = \frac{1}{\cos^2x} = \sec^2 x$\\\\
$\frac{d}{dx} (\tan f(x)) = \frac{1}{\cos^2(f(x))}$\\\\
$\frac{d}{dx} (a^x) = a^x \cdot \ln a$\\\\
$\frac{d}{dx} (a^{f(x)}) = a^{f(x)} \cdot \ln a \cdot f'(x)$\\\\
$\frac{d}{dx} (\log_ax) = \frac{1}{\ln a \cdot x} $\\\\
$\frac{d}{dx} (\log_af(x)) = \frac{1}{\ln a \cdot x} \cdot f'(x) $\\\\


\section{Integration}

\subsection{Power rule}
Example: $\int x^3 dx$\\
The rule: $\int x^n dx = \frac{x^{n+1}}{n+1} + C$\\
Result: $\int x^3 dx = \frac{x^4}{4} + C$

\subsection{Multiplication by constant}
Example: $\int 6x^2 dx$\\
The rule: $\int nx^n dx = n \int x^n dx$\\
Result: $\int 6x^2 dx = 6 \int x^2 dx = 6 \frac{x^3}{3} + C$\\ 

\subsection{Sum rule}
Example: $\int \cos x + x dx$\\
Result: $\int \cos x + x dx = \int \cos x dx + \int x dx$\\

\subsection{Difference rule}
Example: $\int \cos x - x dx$\\
Result: $\int \cos x - x dx = \int \cos x dx - \int x dx$\\

\subsection{Integration by Parts}
The rule: $\int u v dx = u \int v dx - \int u' (\int v dx) dx$




\end{document}