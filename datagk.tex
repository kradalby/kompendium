\documentclass[a4paper, 10pt]{article}
\usepackage[utf8x]{inputenc}
\usepackage[norsk]{babel}
\usepackage{natbib}
\usepackage{graphicx}
\usepackage[T1]{fontenc}
\usepackage{algpseudocode}


\title{Datamaskiner Grunnkurs}
\author{Kristoffer Dalby}
\date{}


\begin{document}

\maketitle

\thispagestyle{empty}
\newpage
\pagenumbering{arabic}
\setcounter{page}{1}

\section{Registre}
\subsection{PC}
Program counter peker på den instruksjonen i programmet som blir jobbet med.

\subsection{MPC}
Micro Program Counter peker på microinstruksjonen i 'control store'.  Den peker på den første micro instruksjonen i den aktive instruksjonen. MPC holder orden på hva den neste micro instruksjonen er.

\subsection{MBR}
Memory Buffer Register er et bufferregister mellom minne og prosessor.

\subsection{IR}
Instruction Register er der kontrollenheten lagrer instruksjonen som blir gjennomført nå. Den ligger her mens instruksjonen blir dekodet, startet og gjennomført.

\subsection{MAR}
Memory Address Register inneholder adresse til neste minnelokasjon der vi finner neste instruksjon.

\subsection{MDR}
Memory Data Register inneholder data som skal bli lagret i hovedminne/RAM, eller data som har blitt hentet fra minne. Det fungerer som en buffer så data er klar for prosessoren.

\subsection{LV}
Local Variable inneholder pekerverdi.

\subsection{SP}
Stack Pointer inneholder pekerverdi mot element i stacken.

\subsection{CPP}
Constant Pool Pointer inneholder pekerverdi.

\subsection{TOS}
Top Of Stack skal alltid inneholde elementet på toppen av stakken

\subsection{OPC}
OpCode register kan fritt brukes.

\subsection{H}
Holding Register inneholder verdien som skal inn i A-inngangen til ALU.

\section{ALU-flagg}
\begin{tabular}{|l|l|}
    \hline
    Flagg & Årsak                                        \\ \hline
    N     & Når svar er negativt                         \\ \hline
    Z     & Når svaret er 0                              \\ \hline
    C     & Carry, når vi legger sammen uten fortegnsbit \\ \hline
    V     & Overflow                                     \\ \hline
\end{tabular}

\section{Von Neumann}
Data og instruksjoner er lagret i samme minne og beregninger skjer sekvensielt. Von Neumann-arkitektur gjør det mulig å skrive program som kan endre sin egen programkode.

\begin{itemize}
	\item ALU - Aritmetisk og logisk enhet som utfører beregningene.
	\item Control Unit - Kontrollenhet som dekoder instruksjoner og gjennomfører dem. \\ Kontrollenheten kan enten være hard-wired eller inneholde kode styrt av en mikrokontroller.
	\item Memory - Primærminnet (RAM) inneholder data og instruksjoner.
	\item I/O - Enheter for inn- og utdata.
	
\end{itemize}

Moderne datamaskiner har ALU og kontrollenhet på prosessor (CPU), benytter seg av registre, hurtigbuffere, busser og millioner av transistorer, men konseptet er veldig likt. Overføring av data mellom minne og CPU blir i dag sett på som et av de største problemene med Von Neumann-arkitektur.

\section{MicroInstruction Register}
\begin{itemize}
	\item Addr - Peker på neste mikroinstruksjon i instruksjonen.
	\item J - Jam sier ifra om ALU har flagget neste mikroinstruksjon eller om det kommer hopp (betinget hopp).
	\item ALU - bestemmer hvilken funksjon ALU skal gjennomføre.
	\item C - Inneholder adressen til C-bussen, som blir det samme som adressen til registeret det skal skrives til. 
	\item Mem - Sier ifra om det skal gjøres noe med minne.
	\item B - inneholder adressen til B-bussen, som blir det samme som adressen til registeret det skal leses fra.
\end{itemize}

\section{Superskalar CPU}
En superskalar prosessor implementerer en form for parallellitet som kalles instruksjonsnivåparallellitet. Dette betyr at den kan utføre flere instruksjoner pr. klokkesyklus (dupliserer CPU-enheter).

\section{Lokalitet}
\subsection{Tid}
Om vi leste fra en minneadresse er det sannsynlig at vi snart vil lese fra den samme adressen igjen.

\subsection{Rom}
Om vi leste fra en minneadresse er det sannsynlig at vi snart vil lese fra naboadressen.

\section{Dataavhengighet}
\subsection{RAW}
Read-After-Write (sanne dataavhengigheter er når for eksempel instruksjon 1 skriver til et register og instruksjon 2 skal lese fra det samme registeret.

\subsection{WAW}
Write-After-Write (utavhengigheter) er når for eksempel instruksjon 3 skriver til register 1 og instruksjon 1 skriver til register 1.

\subsection{WAR}
Write-After-Read (antiavhengigheter er når for eksempel instruksjon 3 skriver til register 1 og instruksjon 2 leser fra register 1.



\section{RAM}
\subsection{SRAM}
Statisk RAM er raskt og trenger ikke opdateres. Brukes ofte i hurtigbuffere.

\subsection{DRAM}
Dynamisk RAM må friskes opp jevnlig. Det tar mindre plass en SRAM (2 vs. 6 transistorer).

\subsection{SDRAM} 
Synkront Dynamisk RAM betyr at data blir overført til/fra RAM synkront med klokka (og systembussen).

\section{CMP}
Chip-level Multiprosessor er flere prosessorer på samme brukke. Bruker samme hurtigbuffer.

\begin{itemize}
	\item Homogene kjerner - alle kjerner er like
	\item Hetrogene kjerner - forskjellige kjerner til forskjellige oppgaver.
\end{itemize}

Fordeler med CMP er lavere effekt/varmeutvikling, bedre utnyttelse av prosessorkraft, mulighet for "ut av rekkefølge" og lettere å utnytte instruksjonsnivåparallellitet.

\section{Adressering}
Måten instruksjonen angir hvor data skal hentes fra kalles en adresseringsmodus.

\begin{itemize}
	\item Immidiate - Operanden er innbakt i instruksjonen. Må være kjent når programmet lages.
	\item Direkte - Instruksjonen angir adressen til operand i RAM.
	\item Indirekte - Instruksjonen angir adresse til RAM-celle som igjen inneholder adressen til operand.
	\item Register - Instruksjonen har nummer på register som inneholder operand.
	\item Indirekte register - Instruksjon har nummer på register som inneholder adressen til operand i RAM.
	\item Stakk - Adressen er implisitt gitt av stakkpeker.
\end{itemize}


\section{Branch Prediction}
Forgreningspredikering

\begin{itemize}
	\item Statisk - Forutsier hopp uavhengig av hvor hopp har forekommet før.
	\item Dynamisk - Forutsier hopp ut i fra hvor det har skjedd hopp før.
\end{itemize}






















\end{document}


